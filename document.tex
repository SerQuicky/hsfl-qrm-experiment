\documentclass{hsflensburg}

% packages
%-------------------------------------------
\usepackage[utf8]{inputenc}
\usepackage[ngerman]{babel}
\usepackage{csquotes}
\usepackage{placeins}
\usepackage{url}
\usepackage{xcolor}

\MakeOuterQuote{"}
{\renewcommand{\arraystretch}{2.0}% for the vertical padding

\definecolor{orange}{HTML}{FF7F00}

% document title
%-------------------------------------------
\title{Experiment - Programmiersprachen}
\subtitle{Qualitative Research Methods}

% authors
%-------------------------------------------
\author{
	\name{Michael Frank}\\
	\institution{Hochschule Flensburg}
}

\begin{document}
	\maketitle
 	 \tableofcontents

  \pagebreak
	
	\section{Motivation}
	Ein Thema, welches unter Softwareentwicklern häufig und ausgiebig diskutiert wird, ist die Qualität von Programmiersprachen. 
	In dieser Diskussion geht es oft darum Programmiersprachen miteinander zu vergleichen und damit zu zeigen, warum 
	\textit{Programmiersprache A} besser als \textit{Programmiersprache B} ist. Dass dieses Thema Softwareentwickler stark beschäftig,
	zeigt sich auch in der jährlichen Softwareentwickler-Umfrage der Internetplattform \textit{Stackoverflow} 
	\cite{stack2019, stack2020}. Bei dieser können Entwickler angeben welche Programmiersprachen sie mögen und welche nicht.
	Eine Frage die sich dabei stellt, welche aber aufgrund der Umfrageart nicht geklärt wird, ist warum die Programmierer bestimmte 
	Programmiersprachen nicht mögen. Dabei ist es natürlich interessant zu wissen, ob die Abneigung gegenüber
	einer Programmiersprache objektive oder subjektive Gründe hat. Diese Frage soll im Rahmen einer Studie untersucht
	werden. In dieser Studie sollen Semi-Strukturierte Interviews mit Softwareentwicklern geführt und herausgefunden
	werden, welche Programmiersprachen Sie mögen und welche nicht und was die Gründe dafür sind. Außerdem
	soll untersucht werden, was für die Entwickler eine gute Programmiersprache ausmacht, um möglicherweise
	einen Funtkionskatalog für Programmiersprachen zusammenfassen zu könne, der die Präferenzen vieler Programmierer
	vereint.


	\clearpage
	\nocite{*}	
	\bibliography{literature}
	\bibliographystyle{abbrv}
\end{document}
